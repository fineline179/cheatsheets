%% LaTeX structure for cheatsheets
%% (compile with XeTeX. see below if you dont have XeTeX.)

\usepackage[utf8]{inputenc}
\usepackage{jheppub_ss}
\usepackage{amsmath, amssymb}
\usepackage{graphicx}
\usepackage{caption}
\usepackage{subcaption}
\usepackage{hyperref}
\usepackage[T1]{fontenc}
\usepackage{lmodern}
% document what this is for
\usepackage{scalerel}
% writes "appendix" on appendices (works in body.. TODO: how to do in contents?)
\usepackage[title]{appendix}
\usepackage{parskip}
\usepackage[backend=biber,style=draft]{biblatex}

% increase spacing between paragraphs (to make up for the lack of indentaiton)
\setlength{\parskip}{4pt}

% added for XITS math font support (uses XeTeX).
% if you don't have XeTeX and XITS font:
%  1) comment out block below
%  2) remove "%%% TeX-engine: xetex" line at end of main file.
% BEGIN NO XeTeX COMMENT BLOCK
\usepackage{unicode-math}
\setmainfont[Path = ./fonts/, Extension = .otf,
             UprightFont = *-Regular, BoldFont = *-Bold, ItalicFont = *-Italic,
             BoldItalicFont = *-BoldItalic]{XITS}
\setmathfont[Path = ./fonts/, Extension = .otf,
             BoldFont = XITSMath-Bold]{XITSMath-Regular}
\setmonofont[Path = ./fonts/, Extension = .ttf, Scale = 0.84]{RobotoMono-Regular}
% \setmonofont[Path = ./fonts/, Extension = .otf, Scale = 0.84] {Monospace 821 Regular}
% END NO XeTeX COMMENT BLOCK

% change sans font to TeX Gyre Heros (Helvetica)
\setsansfont{TeX Gyre Heros}

% for support of bold mathbb letters, see "\mbbb" command below.
\newlength\bshft
\bshft=.18pt\relax
\def\fakebold#1{\ThisStyle{\ooalign{$\SavedStyle#1$\cr%
  \kern-\bshft$\SavedStyle#1$\cr%
  \kern\bshft$\SavedStyle#1$}}}

%%% Macros
\newcommand{\nn}{\nonumber} % kill eqnum
\newcommand{\nncr}{\nonumber\\} % to kill eqnums on all eqs in env except last
\newcommand{\eeg}{\emph{e.g. }}
\newcommand{\eie}{\emph{i.e. }}
%% Text
\newcommand{\tbf}[1]{\textbf{#1}} % bold text
\newcommand{\tit}[1]{\textit{#1}} % italic text
\newcommand{\tbfit}[1]{\textbf{\textit{#1}}} % bold italic text
\newcommand{\ttt}[1]{\texttt{#1}} % verbatim text
%% Math
\newcommand{\trm}[1]{\textrm{#1}} % text stylee math in math env
\newcommand{\mbf}[1]{\mathbf{#1}} % bold math
\newcommand{\mbb}[1]{\mathbb{#1}} % set-type math chars
\newcommand{\mbbb}[1]{\fakebold{\mathbb{#1}}} % set-type bold math chars
%% Bold math SYMBOLS. note \mathbf doesn't work on greek symbols. The modern way to
% handle this is to use the bm package, but it doesn't seem to work.
\newcommand{\bms}[1]{\boldsymbol{#1}}

%% Other
\newcommand{\blue}[1]{{\color{blue}#1}} % blue color shortcut
\newcommand{\TODO}[1]{\blue{\tbf{TODO: #1}}} % colored TODO item
\newcommand{\TODOFIN}[1]{\blue{\tbf{TODO: finish --- #1}}} % colored TODO item++
\renewcommand{\v}{\vec} % abbreviated vector (redefined from \vec)

%% Bold matrix macros
% Latin
\newcommand{\mA}{\mbf{A}}
\newcommand{\mB}{\mbf{B}}
\newcommand{\mC}{\mbf{C}}
\newcommand{\mD}{\mbf{D}}
\newcommand{\mL}{\mbf{L}}
\newcommand{\mM}{\mbf{M}}
\newcommand{\mO}{\mbf{O}}
\newcommand{\mU}{\mbf{U}}
\newcommand{\mV}{\mbf{V}}
\newcommand{\mW}{\mbf{W}}
\newcommand{\mX}{\mbf{X}}
\newcommand{\mY}{\mbf{Y}}
\newcommand{\mZ}{\mbf{Z}}
% Greek
\newcommand{\mSigma}{\bms{\Sigma}}
\newcommand{\mLambda}{\bms{\Lambda}}

% Proclamation styles
\newtheorem{definition}{Definition}

% Non-indenting itemize
\newenvironment{itemize-noindent}
{\setlength{\leftmargini}{0em}\begin{itemize}}
{\end{itemize}}
