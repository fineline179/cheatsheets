%% Prob/Stats cheat sheet
%% (compile with XeTeX. see notes in structure_cheatSheet.tex if no XeTeX.)

\documentclass[11pt]{article}
\synctex=1
%% LaTeX structure for cheatsheets
%% (compile with XeTeX. see below if you dont have XeTeX.)

\usepackage[utf8]{inputenc}
\usepackage{jheppub}
\usepackage{amsmath}
\usepackage{graphicx}
\usepackage{caption}
\usepackage{subcaption}
\usepackage{hyperref}
\usepackage[T1]{fontenc}
\usepackage{lmodern}
% document what this is for
\usepackage{scalerel}
% writes "appendix" on appendices (works in body.. TODO: how to do in contents?)
\usepackage[title]{appendix}
\usepackage{parskip}

% added for XITS math font support (uses XeTeX).
% if you don't have XeTeX and XITS font:
%  1) comment out block below
%  2) remove "%%% TeX-engine: xetex" line at end of main file.
% BEGIN NO XeTeX COMMENT BLOCK
\usepackage{unicode-math}
\setmainfont[Extension = .otf, UprightFont = *-regular, BoldFont = *-bold,
             ItalicFont = *-italic, BoldItalicFont = *-bolditalic]{xits}
\setmathfont[Extension = .otf, BoldFont = *bold]{xits-math}
\setmonofont[Path = ./fonts/, Extension = .ttf, Scale = 0.84]
{RobotoMono-Regular}
% \setmonofont[Path = ./fonts/, Extension = .otf, Scale = 0.84]
% {Monospace 821 Regular}
% END NO XeTeX COMMENT BLOCK

% for support of bold mathbb letters, see "\mbbb" command below.
\newlength\bshft
\bshft=.18pt\relax
\def\fakebold#1{\ThisStyle{\ooalign{$\SavedStyle#1$\cr%
  \kern-\bshft$\SavedStyle#1$\cr%
  \kern\bshft$\SavedStyle#1$}}}

%% macros
\newcommand{\nn}{\nonumber} % kill eqnum
\newcommand{\nncr}{\nonumber\\} % to kill eqnums on all eqs in env except last
\newcommand{\eeg}{\emph{e.g. }}
\newcommand{\eie}{\emph{i.e. }}
% text
\newcommand{\tbf}[1]{\textbf{#1}} % bold text
\newcommand{\tit}[1]{\textit{#1}} % italic text
\newcommand{\tbfit}[1]{\textbf{\textit{#1}}} % bold italic text
\newcommand{\ttt}[1]{\texttt{#1}} % verbatim text
% math
\newcommand{\trm}[1]{\textrm{#1}} % text stylee math in math env
\newcommand{\mbf}[1]{\mathbf{#1}} % bold math
\newcommand{\mbb}[1]{\mathbb{#1}} % set-type math chars
\newcommand{\mbbb}[1]{\fakebold{\mathbb{#1}}} % set-type bold math chars
% other
\newcommand{\blue}[1]{{\color{blue}#1}} % blue color shortcut
\newcommand{\TODO}[1]{\blue{\tbf{TODO: #1}}} % colored TODO item
\newcommand{\TODOFIN}[1]{\blue{\tbf{TODO: finish --- #1}}} % colored TODO item++


% proclamation styles
\newtheorem{definition}{Definition}

% Non-indenting itemize
\newenvironment{itemize-noindent}
{\setlength{\leftmargini}{0em}\begin{itemize}}
{\end{itemize}}

% increase spacing between paragraphs (to make up for the lack of indentaiton)
\setlength{\parskip}{4pt}

\usepackage{xfrac}
\addbibresource{refs.bib}

%%%%%%%%%%%%%%%%%%%%%%%%%%%%%%%%%%%%%%%%%%%%%%%%%%%%%%%%%%%%%%%%%%%%%%%%%%%%%%%%%%%%%%%%
\title{Prob/Stats Cheatsheet}
\author{Steve Young}
\abstract{Everything I know about prob/stats/maybe information theory too..}

\begin{document}
\maketitle

%%%%%%%%%%%%%%%%%%%%%%%%%%%%%%%%%%%%%%%%%%%%%%%%%%%%%%%%%%%%%%%%%%%%%%%%%%%%%%%%%%%%%%%%
%%%%%%%%%%%%%%%%%%%%%%%%%%%%%%%%%%%%%%%%%%%%%%%%%%%%%%%%%%%%%%%%%%%%%%%%%%%%%%%%%%%%%%%%
\section{Conventions}
\paragraph{Math Notation}



%%%%%%%%%%%%%%%%%%%%%%%%%%%%%%%%%%%%%%%%%%%%%%%%%%%%%%%%%%%%%%%%%%%%%%%%%%%%%%%%%%%%%%%%
\section{Distributions}
\subsection{Gaussians}
\subsubsection{Basics}
\begin{enumerate}
  \item To start with, \emph{memorize} that
  \begin{equation}
    \boxed{\int_{-\infty}^{\infty} dx\, e^{-x^2} = \pi^{1/2}}
  \end{equation}

  \item Next, anything multiplying the $x^2$ in the integrand is present in inverse
  under the square root.
  \begin{equation}
    \int_{-\infty}^{\infty} dx\, e^{-\trm{stuff}\, x^2} =
    \lp \frac{\pi}{\trm{stuff}} \rp^{1/2}
  \end{equation}
  so, for example:
  \begin{equation}
    \label{eq:int_gauss_a2}
    \int_{-\infty}^{\infty} dx\, e^{-\frac{1}{2} a x^2} = \lp \frac{2 \pi}{a} \rp^{1/2} 
  \end{equation}

  \item The traditional Gaussian pdf is thus easily seen to be
  \begin{equation}
    \mathcal{N}(x|0,\sigma^2) = \frac{1}{\lp 2 \pi \sigma^2 \rp^{1/2}}\,
    e^{-\frac{1}{2 \sigma^2} x^2}
  \end{equation}
\end{enumerate}

\subsubsection{Differentiation moment trick}
By differentiating Eq. \ref{eq:int_gauss_a2} wrt $a$, we obtain an expression for
integrals of the form $\int_{-\infty}^{\infty}dx \, x^{2n} e^{-\frac{1}{2} a x^2}$, with
$n \in \mathbb{Z^+}$.

\eeg for $n=1$:
\begin{equation}
  - 2 \frac{d}{da} \int_{-\infty}^{\infty}dx \, e^{-\frac{1}{2} a x^2} =
  \int_{-\infty}^{\infty}dx \, x^2 e^{-\frac{1}{2} a x^2} =
  - 2 \frac{d}{da} \lp \frac{2\pi}{a} \rp^{1/2} =
  \lp \frac{2\pi}{a} \rp^{1/2} \, \frac{1}{a} 
\end{equation}
For $n=2$:
\begin{equation}
  \lp -2 \frac{d}{da} \rp^2 \int_{-\infty}^{\infty}dx \, e^{-\frac{1}{2} a x^2} =
  \int_{-\infty}^{\infty}dx \, x^4 e^{-\frac{1}{2} a x^2} =
  \lp -2 \frac{d}{da} \rp^2 \lp \frac{2\pi}{a} \rp^{1/2} =
  \lp \frac{2\pi}{a} \rp^{1/2} \, \frac{1}{a} \, \frac{3}{a}
\end{equation}
We thus obtain an expression for the expectation value of $x^{2n}$ under the Gaussian
distribution: 
\begin{equation}
  \langle x^{2n} \rangle =
  \frac{\int_{-\infty}^{\infty}dx \, x^{2n} e^{-\frac{1}{2} a x^2}}
       {\int_{-\infty}^{\infty}dx \, e^{-\frac{1}{2} a x^2}} =
  \frac{1}{a^n} (2n-1) (2n-3) \cdot\cdot\cdot 5 \cdot 3 \cdot 1
\end{equation}

\subsubsection{Gaussian with Linear Term}
To evaluate integrals of the form
\begin{equation}
  \label{eq:int_gauss_a2_j}
  \int_{-\infty}^{\infty} dx\, e^{-\frac{1}{2} a x^2 + J x},
\end{equation}
first complete the square in the exponent
\begin{equation}
  -\frac{a}{2} x^2 + J x = -\frac{a}{2} (x^2 - \frac{2Jx}{a}) =
  -\frac{a}{2} \lp x - \frac{J}{a} \rp^2 + \frac{J^2}{2a}
\end{equation}
which gives
\begin{equation}
  \int_{-\infty}^{\infty} dx\, e^{-\frac{1}{2} a x^2 + J x} =
  \int_{-\infty}^{\infty} dx\, e^{-\frac{1}{2} a\, \lp x - J/a \rp} e^{J^2 / 2a} =
  \lp\frac{2\pi}{a} \rp^{1/2} \, e^{J^2 / 2a}
\end{equation}
where the first integral is done by shifting $x \to x + J a$.

Eq. \ref{eq:int_gauss_a2_j} is also (close to) the \tbfit{moment generating function} of
the Gaussian distribution. Given a pdf $p(x)$, the moment generating function is defined
as
\begin{equation}
  \psi_x(J) = \int_{-\infty}^{\infty} dx\, e^{J x} p(x)
\end{equation}
so that the moment generating function for the Gaussian distribution is
\begin{equation}
  \frac{1}{\lp 2 \pi \sigma^2 \rp^{1/2}}\, \int_{-\infty}^{\infty} dx\,
  e^{J x}  e^{-\frac{1}{2 \sigma^2} x^2}, 
\end{equation}
\TODO{Finish. Figure out clear way to include normalization factor of pdf in exposition}

\subsubsection{Multivariate Gaussians}
Promoting $a$ to a real $N \times N$ symmetric matrix $\mA$, and x and J to a $N$-dim
vectors $\v{x}$ and $\v{J}$ with components $x_i$ and $J_i$, we have the multivariate
Gaussian integral 
\begin{equation}
  \prod_{i=1}^N \lp \int_{-\infty}^{\infty} dx_i \rp
  e^{-\frac{1}{2} \v{x}^T \mA \v{x} + \v{J}^T \v{x}} =
  \lp \frac{(2 \pi)^N}{|\mA|} \rp^{1/2} e^{\frac{1}{2} \v{J}^T \mA^{-1} \v{J}}
\end{equation}
\TODOFIN{}

\subsection{Bernoulli}
For $x \in \{0, 1\}$, Bernoulli dist parametrized by $\mu$, with
\begin{equation}
  p(x; \mu) = \mu^x (1-\mu)^{1-x}
\end{equation}


%%%%%%%%%%%%%%%%%%%%%%%%%%%%%%%%%%%%%%%%%%%%%%%%%%%%%%%%%%%%%%%%%%%%%%%%%%%%%%%%%%%%%%%%
\section{Prob and stats}
\subsection{The Rules of Probability}
\begin{itemize}
  \item \tbf{Product Rule}: $p(x, y) = p(x|y) p(y) = p(y|x) p(x)$
  \item \tbf{Sum Rule}: $p(x) = \sum\limits_y p(x, y) = \sum\limits_y p(x | y) p(y)$
\end{itemize}

\subsection{Bayes' Rule}
Using $p(y|x) p(x) = p(x,y) = p(x|y) p(y)$, we have
\begin{equation}
  p(y|x) = \frac{p(x|y) p(y)}{p(x)} = \frac{p(x|y) p(y)}{\sum\limits_y p(x|y) p(y)}
\end{equation}

\subsection{Expectation and Variance}
\label{sec:ExpVar}
\begin{itemize}
  \item \tbf{Expectations of sum of variables add}:\\
  If $X_1, \dots, X_n$ are random variables, and $a_1, \dots, a_n$ are constants, then
  \begin{equation}
    \E \lp \sum_{i=1}^n a_i X_i \rp = \sum_{i=1}^n a_i\, \E (X_i)
  \end{equation}

  \item \tbf{Variances of sum of independent variables add}:\\
  If $X_1, \dots, X_n$ are \emph{independent} random variables, and $a_1, \dots, a_n$
  are constants, then
  \begin{equation}
    \trm{Var} \lp \sum_{i=1}^n a_i X_i \rp = \sum_{i=1}^n a_i^2\, \trm{Var} (X_i)
  \end{equation}

  \item \tbf{Variances of sum of (dependent) variables}:\\
  If $X$ and $Y$ are random variables, then
  \begin{align}
    \trm{Var} (X + Y) &= \trm{Var}(X) + \trm{Var}(Y) + 2 \trm{Cov}(X, Y) \nncr
    \trm{Var} (X - Y) &= \trm{Var}(X) + \trm{Var}(Y) - 2 \trm{Cov}(X, Y)
  \end{align}
  
\end{itemize}

\subsection{Central Limit Theorem}
Let $S_N = \sum_{i=1}^N X_i$, where each $X_i$ is \tbf{iid} with mean $\mu$ and variance
$\sigma^2$. Then, as $N \to \infty$, the pdf of $S_N$ approaches a normal distribution:
\begin{equation}
  p(S_N = s) = \frac{1}{\lp 2 \pi N \sigma^2 \rp^{1/2}} \exp \lb - \frac{(s - N
    \mu)^2}{2 N \sigma^2} \rb 
\end{equation}
NB the factors of $N$ in the pdf, which make the pdf mean/variance equal to $N$ times
the original mean/variance (\eie means and variances of independent variables add; see
section \ref{sec:ExpVar}.)


\subsection{(Weak) Law of Large Numbers}
Let $X_1, \dots, X_n$ be \tbf{iid}, and $\mu = \E(X_1)$\footnote{$\mu = \E(X_1) =
  \E(X_i)$ for any $1 \leq i \leq n$}. Defining the \tbfit{sample mean} as
$\overline{X}_n = n^{-1} \sum_{i=1}^n X_i$, the WLLN states that $\overline{X}_n$
converges in probability to $\mu$. 

%%%%%%%%%%%%%%%%%%%%%%%%%%%%%%%%%%%%%%%%%%%%%%%%%%%%%%%%%%%%%%%%%%%%%%%%%%%%%%%%%%%%%%%%
\section{Information Theory}
\paragraph{KL divergence:}
\begin{align}
  KL\big[p(x) || q(x)\big] &= \sum_{x_i} p(x_i) \log \lp \frac{p(x_i)}{q(x_i)} \rp
                             = -\sum_{x_i} p(x_i) \log \lp \frac{q(x_i)}{p(x_i)} \rp
                             \nncr 
                           &= -\sum_{x_i} p(x_i) \log q(x_i) + \sum_{x_i} p(x_i) \log
                             p(x_i) \nncr 
                           &= H(p,q) - H(p)
\end{align}
where $H(p,q)$ is the cross entropy, and $H(p)$ is the entropy.


%%%%%%%%%%%%%%%%%%%%%%%%%%%%%%%%%%%%%%%%%%%%%%%%%%%%%%%%%%%%%%%%%%%%%%%%%%%%%%%%%%%%%%%%
\section{Bayesian}



%%%%%%%%%%%%%%%%%%%%%%%%%%%%%%%%%%%%%%%%%%%%%%%%%%%%%%%%%%%%%%%%%%%%%%%%%%%%%%%%%%%%%%%%
\section{Optimal Stopping Theory}



%%%%%%%%%%%%%%%%%%%%%%%%%%%%%%%%%%%%%%%%%%%%%%%%%%%%%%%%%%%%%%%%%%%%%%%%%%%%%%%%%%%%%%%%
%%%%%%%%%%%%%%%%%%%%%%%%%%%%%%%%%%%%%%%%%%%%%%%%%%%%%%%%%%%%%%%%%%%%%%%%%%%%%%%%%%%%%%%%
\newpage
\printbibliography[heading=bibintoc]

%%%%%%%%%%%%%%%%%%%%%%%%%%%%%%%%%%%%%%%%%%%%%%%%%%%%%%%%%%%%%%%%%%%%%%%%%%%%%%%%%%%%%%%%
%%%%%%%%%%%%%%%%%%%%%%%%%%%%%%%%%%%%%%%%%%%%%%%%%%%%%%%%%%%%%%%%%%%%%%%%%%%%%%%%%%%%%%%%
\end{document}

%%% Local Variables:
%%% mode: latex
%%% TeX-master: t
%%% TeX-engine: xetex
%%% End:
